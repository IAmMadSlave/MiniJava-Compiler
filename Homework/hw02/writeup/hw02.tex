%----------------------------------------------------------------------------------------
%	PACKAGES AND OTHER DOCUMENT CONFIGURATIONS
%----------------------------------------------------------------------------------------

\documentclass{article}

\usepackage{fancyhdr} % Required for custom headers
\usepackage{lastpage} % Required to determine the last page for the footer
\usepackage{extramarks} % Required for headers and footers
\usepackage{graphicx} % Required to insert images
\usepackage{listings}
\usepackage{color}
\usepackage{xcolor}
\usepackage{caption}
\usepackage{enumitem}
\usepackage{amsmath}
\usepackage{tikz}
\usetikzlibrary{calc,shapes.multipart,chains,arrows}

\DeclareCaptionFont{white}{\color{white}}
\DeclareCaptionFormat{listing}{%
\parbox{\textwidth}{\colorbox{gray}{\parbox{\textwidth}{#1#2#3}}\vskip-4pt}}
\captionsetup[lstlisting]{format=listing,labelfont=white,textfont=white}

% Margins
\topmargin=-0.45in
\evensidemargin=0in
\oddsidemargin=0in
\textwidth=6.5in
\textheight=9.0in
\headsep=0.25in 

\linespread{1.1} % Line spacing

% Set up the header and footer
\pagestyle{fancy}
\lhead{\hmwkAuthorName} % Top left header
\chead{\hmwkClass\ (\hmwkClassInstructor\ \hmwkClassTime): \hmwkTitle} % Top center header
\rhead{\hmwkDueDate} % Top right header
\lfoot{\lastxmark} % Bottom left footer
\cfoot{} % Bottom center footer
\rfoot{Page\ \thepage\ of\ \pageref{LastPage}} % Bottom right footer
\renewcommand\headrulewidth{0.4pt} % Size of the header rule
\renewcommand\footrulewidth{0.4pt} % Size of the footer rule

\setlength\parindent{0pt} % Removes all indentation from paragraphs

%----------------------------------------------------------------------------------------
%	DOCUMENT STRUCTURE COMMANDS
%	Skip this unless you know what you're doing
%----------------------------------------------------------------------------------------

\setcounter{secnumdepth}{0} % Removes default section numbers
\newcounter{homeworkProblemCounter} % Creates a counter to keep track of the number of problems

\newcommand{\homeworkProblemName}{}
\newenvironment{homeworkProblem}[1][Problem \arabic{homeworkProblemCounter}]{ % Makes a new environment called homeworkProblem which takes 1 argument (custom name) but the default is "Problem #"
\stepcounter{homeworkProblemCounter} % Increase counter for number of problems
\renewcommand{\homeworkProblemName}{#1} % Assign \homeworkProblemName the name of the problem
\section{\homeworkProblemName} % Make a section in the document with the custom problem count
}

%----------------------------------------------------------------------------------------
%   COLORS AND LANGUAGAGE
%----------------------------------------------------------------------------------------

\lstset{
    frame=lrb,xleftmargin=\fboxsep,xrightmargin=-\fboxsep,language=Java,basicstyle=\ttfamily,
    breaklines=true,columns=fullflexible,keepspaces=true,escapeinside={\%*}{*)}
       }

%----------------------------------------------------------------------------------------
%	NAME AND CLASS SECTION
%----------------------------------------------------------------------------------------

\newcommand{\hmwkTitle}{Homework\ \#2} % Assignment title
\newcommand{\hmwkDueDate}{Wednesday,\ Feburary\ 11,\ 2015} % Due date
\newcommand{\hmwkClass}{COP\ 5621} % Course/class
\newcommand{\hmwkClassTime}{6:25pm} % Class/lecture time
\newcommand{\hmwkClassInstructor}{Smith} % Teacher/lecturer
\newcommand{\hmwkAuthorName}{Musa V. Ahmed \& Jose Acosta} % Your name

%----------------------------------------------------------------------------------------

\begin{document}
\belowcaptionskip=-10pt

%----------------------------------------------------------------------------------------
%	PROBLEM 1
%----------------------------------------------------------------------------------------

\begin{homeworkProblem}
    In this assignment, we begin the compiler project by implementing a lexical analyzer for
    MiniJava. To find out details of MiniJava's tokens, you will need to refer to the Appendix
    on pages 484-486.

    \vspace{2 mm}
    We will not be using the tools JavaCC or SableCC which are discussed (briefly!) in Chapter
    2. Instead we will be using tools called JLex and CIP; these are Java versions of the
    classic Unix tools, lex and yacc. You can find the JLex Manual at the course Moodel page.

    \vspace{2 mm}
    As mentioned in the syllabus, you need to set some UNIX environment variables to access
    JLex. Add the following lines to your \verb!.cshr! files:

    \begin{lstlisting}[frame=none]
    setenv JAVA_HOME /depot/JSE-1.8
    setenv CLASSPATH "/homes/smithg/compiler:."
    \end{lstlisting}


    \vspace{-3 mm}
    My \verb!/homes/smithg/compiler/minijava/chap2! directory contains a number of files to
    help you; you can copy them to your directory with the command

    \begin{lstlisting}[frame=none]
    cp -rp /homes/smithg/compiler/minijava/chap2 .
    \end{lstlisting}

    \vspace{-3 mm}
    The only file you need to modify is \verb!parse/MiniJava.lex!, which contains the JLex
    specification. Once you have completed \verb!MiniJava.lex!, you can run JLex and compile
    everything simply by typing 

    \begin{lstlisting}[frame=none]
    make
    \end{lstlisting}

    \vspace{-3 mm}
    from \verb!chap2!; and then you can run your lexer by typing

    \begin{lstlisting}[frame=none]
    java parse.Main test.java
    \end{lstlisting}

    \vspace{-3 mm}
    Please upload your \verb!MiniJava.lex! file to the Moodle web site, and also please prepare
    a hard copy to submit in class. As always, it would be nice to include printouts of some
    sample executions.

    Here are a few more tips:
    
    \begin{itemize}[noitemsep]
        \item You should treat everything written in \textbf{boldface} in the grammar on page
            485 as a reserved word. You may also wish to look at \verb!parse/sym.java!, which
            defines constants for all of the token that you must return. (It also defines a
            constant for \verb!error!, but this is not a real token; it is used by CUP for
            error recovery.) Note also that you should not recognize \textbf{extends}, which is
            supported only when MiniJava is augmented with inheritance in Chapter 14.
        \item Contrary to what is said on page 484, Java /* */ comments may \emph{not} be
            nested.
        \item To interface with CUP, your lexer must return token with class
            \verb!java_cup/runtime\Symbol!. You can create such objects by calling the class
            constructor or by using the handy method \verb!tok! defined in \verb!MiniJava.lex!.
            Notice that a \verb!Symbol! has a \verb!value! field of class \verb!Object!, which
            holds the semantic value of the token. For most MiniJava tokens, \verb!value! should be
            null, but for an \verb!ID! it should be a \verb!String!, and for an
            \verb!INTEGER_LITERAL! it should be an \verb!Integer!.
        \item For printing nice error messages, the class \verb!errormsg/ErrorMsg! provides an
            \verb!error! method that takes two parameters: an \verb!int pos!, which tells where
            in the source file the error occured, and a \verb!String msg!. To get error
            positions, using the JLex variable \verb!yychar!, which gives the position of the
            last token matched.
        \item You should use JLex \emph{lexical states} to deal with /* */ comments.
        \item To test your lexer, you can try out the file \verb!test.java! and compare your
            output with mine in \verb!test.out!.
        \item We won't be using CUP until Chapter 3, but some of the code included in the
            \verb!chap2/! skeleton files deals with the interface between JLex and CUP. If you
            are curious about this, you can look at the CUP Manual on the course Moodle page.
        \item This assignment is worth 10 points.
    \end{itemize}
    \clearpage

    \lstinputlisting[language=Java,basicstyle=\ttfamily\scriptsize,frame=none,caption={MiniJava.lex}]{MiniJava.lex}

    \lstinputlisting[language=Java,basicstyle=\ttfamily\scriptsize,frame=none,caption={test.java}]{test.java}
    \lstinputlisting[language=Java,basicstyle=\ttfamily\scriptsize,frame=none,caption={test.out}]{test.out}

    \lstinputlisting[language=Java,basicstyle=\ttfamily\scriptsize,frame=none,caption={test2.java}]{test2.java}
    \lstinputlisting[language=Java,basicstyle=\ttfamily\scriptsize,frame=none,caption={test2.out}]{test2.out}

    \section{Sources}
    JLex Manual\\
    https://www.cs.princeton.edu/~appel/modern/java/JLex/current/manual.html \\

    JFlex Manual was also helpful\\
    http://jflex.de/manual.html \\


\end{homeworkProblem}
\clearpage

%----------------------------------------------------------------------------------------

\end{document}
